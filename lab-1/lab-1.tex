\documentclass[12pt]{article}  
\usepackage{ucs} 
\usepackage[utf8x]{inputenc} % Включаем поддержку UTF8  
\usepackage[russian]{babel}  % Включаем пакет для поддержки русского языка 
\usepackage{amsmath}

\begin{document}
\title{Вариант 12}
\date{}
\maketitle
\section{Задание 1. Привести уравнение к каноническому виду}
$$u_{xx} + 4u_{xy} + 13u_{yy} + 3u_x - 9u = -9(x+y)$$
Характеристическое уравнение:
$$(dy)^2 - 4dxdy + 13(dx)^2 = 0$$
$$D = 16(dx)^2 - 52(dx)^2 = -36(dx)^2 < 0$$
Значит, уравнение элиптического типа
$$dy = \frac{4 \pm 6i}{2}dx \Rightarrow y - \frac{4 \pm 6i}{2}x = C$$
$$\xi = y - 2x, \xi_x = -2, \xi_y = 1$$
$$\eta = 3x, \eta_x = 3, \eta_y = 0$$

$$u_x = u_{\xi}\xi_x + u_{\eta}\eta_x = -2u_{\xi} + 3u_{\eta}$$
$$u_y = u_{\xi}\xi_y + u_{\eta}\eta_y = u_{\xi}$$
$$u_{xx} = -2u_{\xi\xi}\xi_x - 2u_{\xi\eta}\eta_x + 3u_{\xi\eta}\xi_x + 3u_{\eta\eta}\eta_x 
= 4u_{\xi\xi} - 12u_{\xi\eta}+9u_{\eta\eta}$$
$$u_{xy} = (u_y)_x = u_{\xi\xi}\xi_x + u_{\xi_\eta}\eta_x
= -2u_{\xi\xi} + 3u_{\xi_\eta}$$
$$u_{yy} = u_{\xi\xi}\xi_y + u_{\xi_\eta}\eta_y 
= u_{\xi\xi}$$
Вычислим коэффициенты:
$$u_{\xi\xi}: 4 - 8 + 13 = 9$$
$$u_{\xi\eta}: -12 + 12 = 0$$
$$u_{\eta\eta}: 9$$
$$u_{\xi} = -6$$
$$u_{\eta} = 9$$
Имеем
$$9u_{\xi\xi} + u_{\eta\eta} - 6u_{\xi} +9u_{\eta} - 9u = -9(\xi+\eta)$$
$$$$

\section{Задание 2. Привести уравнение к каноническому виду}
$$y^2u_{xx}+2xu_{xy} + 2x^2u_{yy} + yu_y = 0$$

Характеристическое уравнение:
$$y^2(dy)^2 - 2xdxdy + 2x^2(dx)^2 = 0$$
$$D = (4x^2 - 8x^2y^2)(dx)^2 = 4x^2(1 - 2y^2)(dx)^2$$

\subsection{Случай 1. Уравнение гиперболического типа} 

$$D > 0 \Rightarrow 1 - 2y^2 > 0 \Rightarrow y^2 < 1/2$$
$$dy = \frac{2x + 2x\sqrt{1-2y^2}}{2y^2}dx = \frac{1+\sqrt{1-2y^2}}{y^2}xdx$$
$$\xi = 1/2y - 1/2x^2 - 1/2 \int\limits_{0}^{y} \sqrt{1-2t^2} \,dt$$
$$\xi_x = -x$$ 
$$\xi_y = 1/2(1 - \sqrt(1-2y^2))$$
$$\eta = 1/2y - 1/2x^2 + 1/2 \int\limits_{0}^{y} \sqrt{1-2t^2} \,dt$$
$$\eta_x = -x$$ 
$$\eta_y = 1/2(1 + \sqrt{1-2y^2})$$

$$u_x = u_{\xi}\xi_x + u_{\eta}\eta_x = -xu_{\xi} - xu_{\eta}$$
$$u_y = u_{\xi}\xi_y + u_{\eta}\eta_y = 1/2(1 - \sqrt{1-2y^2})u_{\xi} + 1/2(1 + \sqrt{1-2y^2})u_{\eta}$$

$$u_{xx} = -u_\xi - u_\eta - x(u_{\xi\xi}\xi_x + u_{\xi\eta}\eta_x)
- x(u_{\xi\eta}\xi_x + u_{\eta\eta}\eta_x) = 
x^2u_{\xi\xi} + 2x^2u_{\xi\eta} + x^2u_{\eta\eta} - u_\xi - u_\eta$$

\begin{multline}
u_xy = -x(u_{\xi\xi}\xi_y + u_{\xi\eta}\eta_y + u_{\xi\eta}\xi_y + u_{\eta\eta}\eta_y) \notag \\
= -1/2x(1 - \sqrt{1-2y^2})u_{\xi\xi} - xu_{\xi\eta} - 1/2x(1 + \sqrt{1-2y^2})u_{\eta\eta}
\end{multline}

\begin{align}
    u_yy &= (1/2(1 - \sqrt{1-2y^2})u_\xi + 1/2(1 + \sqrt{1-2y^2})u_\eta)_y \notag \\
    &\quad = 1/2(-y^2 - \sqrt{1-2y^2} + 1)U_{\xi\xi} + y^2u_{\xi\eta} \notag \\ 
    & \quad + 1/2(-y^2 + \sqrt{1-2y^2} + 1)u_{\eta\eta} \notag \\
    &\quad + \frac{y}{\sqrt{1-2y^2}}u_\xi - \frac{y}{\sqrt{1-2y^2}}u_\eta \notag
\end{align}

Вычислим коэффициенты

$$u_{\xi\xi} : x^2y^2 - 1/2(1 - \sqrt{1-2y^2})*2x^2 + 
+ 2x^2*1/2(-y^2 - \sqrt{1-2y^2} + 1) = 0$$

$u_{\xi\eta} : 2x^2y^2 - 2x^2 + 2x^2y^2 = 4x^2y^2 - 2x^2$
$$u_{\eta\eta} : x^2y^2 - 2x^2*1/2(1+\sqrt{1-2y^2}) + 2x^2*1/2(-y^2 + \sqrt{1-2y^2} + 1) = 0$$

$u_\xi : -y^2 + \frac{2x^2y}{\sqrt{1-2y^2}} + y/2(1-\sqrt{1-2y^2})$

$u_\eta : -y^2 - \frac{2x^2y}{\sqrt{1-2y^2}} + y/2(1+\sqrt{1-2y^2})$

Имеем

\begin{align}
    &\quad(4x^2y^2 - 2x^2)u_{\xi\eta} \notag \\
    &\quad+ (-y^2 + \frac{2x^2y}{\sqrt{1-2y^2}} + y/2(1-\sqrt{1-2y^2}))u_\xi \notag \\
    &\quad+ (-y^2 - \frac{2x^2y}{\sqrt{1-2y^2}} + y/2(1+\sqrt{1-2y^2}))u_\eta = 0
\end{align}

\subsection{Случай 2. Уравнение параболического типа}
$D = 4x^2(1-2y^2) = 0, x = 0$
$$dy = -\frac{2x}{2y^2}dx \Rightarrow 1/3y^3 = C$$
$$\xi = 1/3y^3, \xi_x = 0, \xi_y = y^2$$
$$\eta = x, \eta_x = 1, \eta_y = 0$$

$$u_x = u_\xi\xi_x + u_\eta\eta_x = u_\eta$$
$$u_y = u_\xi\xi_y + u_\eta\eta_y = y^2u_\xi$$

$$u_{xx} = u_{\xi\eta}\xi_x + u_{\eta\eta}\eta_x = u_{\eta\eta}$$
$$u_{xy} = u_{\xi\eta}\xi_y + u_{\eta\eta}\eta_y = y^2u_{\xi\eta}$$
$$u_{yy} = 2yu_\xi + y^2u_{\xi\xi}\xi_y + y^2u_{\xi\eta}\eta_y = 2yu_\xi + y^4u_{\xi\xi}$$

Вычислим коэффициенты

\begin{align}
&=u_{\xi\xi} : 2x^2y^4 = 0 \\ \notag
&=u_{\xi\eta}: 2xy^2 = 0 \\ \notag
&=u_{\eta\eta}: y^2 \\ \notag
&=u_\xi: 4x^2y + y^3 \\ \notag
&=u_\eta: 0 \\ \notag
\end{align}

Имеем
$$y^2u_{\eta\eta} + (4x^2y + y^3)u_\xi = 0$$


\subsection{Случай 3. Уравнение элиптического типа}
$$D = 4x^2(1-2y^2) < 0 \Rightarrow y^2 > 1/2$$
$$dy = \frac{1 \pm \sqrt{1-2y^2}i}{y^2}xdx$$

\section{Задание 3. Привести уравнение к каноническому виду в каждой области, где сохраняется тип уравнения}
$$yu_{xx} + xu_{yy} = 0$$
Характеристическое уравнение:
$$y(dy)^2 + x(dx)^2 = 0$$
$$D = -4xy$$

\subsection{Случай 1. Уравнение гиперболического типа}
$$D > 0, x < 0, y > 0$$
$$dy = \sqrt{-\frac{x}{y}}dx,
dy = -\sqrt{-\frac{x}{y}}dx$$

$$\xi = 2/3y^{3/2} + 2/3(-x)^{3/2}, \xi_x = -\sqrt{-x}, \xi_y = \sqrt{y}$$
$$\eta = 2/3y^{3/2} - 2/3(-x)^{3/2}, \eta_x = \sqrt{-x}, \eta_y = \sqrt{y}$$

\begin{align}
u_x &= u_\xi\xi_x + u_\eta\eta_x = -\sqrt{-x}u_\xi + \sqrt{-x}u_\eta \notag \\ \notag
u_y &= u_\xi\xi_y + u_\eta\eta_y = \sqrt{y}u_\xi + \sqrt{y}u_\eta \\ \notag
u_{xx} &= \frac{1}{2\sqrt{-x}}u_\xi - \frac{1}{2\sqrt{-x}}u_\eta 
- \sqrt{-x}(u_{\xi\xi}\xi_x + u_{\xi\eta}\eta_x)
+ \sqrt{-x}(u_{\xi\eta}\xi_x + u_{\eta\eta}\eta_x) \\ \notag
&= -xu_{\xi\xi} + 2xu_{\xi\eta} - xu_{\eta\eta} + \frac{1}{2\sqrt{-x}}u_\xi - \frac{1}{2\sqrt{-x}}u_\eta \\ \notag
u_{xy} &= \sqrt{y}(u_{\xi\xi}\xi_x + u_{\xi\eta}\eta_x) + \sqrt{y}(u_{\xi\eta}\xi_x + u_{\eta\eta}\eta_x)  \\ \notag
&= -\sqrt{-xy}u_{\xi\xi} + \sqrt{-xy}u_{\xi\eta} \\ \notag
u_{yy} &= \frac{1}{2\sqrt{y}}u_\xi + \frac{1}{2\sqrt{y}}u_\eta + \sqrt{x}(u_{\xi\xi}\xi_y + u_{\xi\eta}\eta_y + u_{\xi\eta}\xi_y + u_{\eta\eta}\eta_y) \\ \notag
&= \frac{1}{2\sqrt{y}}u_\xi + \frac{1}{2\sqrt{y}}u_\eta + yu_{\xi\xi} + 2yu_{\xi\eta} + yu_{\eta\eta} \notag
\end{align}

Вычислим коэффициенты
\begin{align}
u_{\xi\xi} &: -xy + xy = 0 \notag \\ \notag
u_{\xi\eta} &: 2xy + 2xy = 4xy \\ \notag
u_{\eta\eta} &: xy - xy = 0 \\ \notag
u_\xi &: -\frac{1}{2\sqrt{-xy}}(\frac{1}{\sqrt{y}} - \frac{1}{\sqrt{-x}}) \\ \notag
u_\eta &: -\frac{1}{2\sqrt{-xy}}(\frac{1}{\sqrt{y}} + \frac{1}{\sqrt{-x}}) \\ \notag
\end{align}

Итог:
\begin{align}
4xyu_{\xi\eta} \notag
- \frac{1}{2\sqrt{-xy}}(\frac{1}{\sqrt{y}} - \frac{1}{\sqrt{-x}})u_\xi \notag 
&- \frac{1}{2\sqrt{-xy}}(\frac{1}{\sqrt{y}} + \frac{1}{\sqrt{-x}})u_\eta = 0 \notag
\end{align}

\subsection{Случай 2. Уравнение параболического типа}
$$D = 0, x = 0 \Rightarrow dy = 0 \Rightarrow y = C$$
$$\xi = y, \xi_x = 0, \xi_y = 1$$
$$\eta = x, \eta_x = 1, \eta_y = 0$$

\begin{align}
u_x &= u_\eta, \notag
u_y = u_\xi, \notag \\
u_{xx} &= u_{\eta\eta}, \notag
u_{xy} = u_{\xi\eta}, \notag
u_{yy} = u_{\xi\xi} \notag
\end{align}

Вычислим коэффициенты
\begin{align}
u_{\xi\xi} &: x = 0 \notag \\
u_{\xi\eta} &: 0 \notag \\
u_{\eta\eta} &: y \notag \\
u_{\xi} &: 0 \notag \\
u_\eta &: 0 \notag
\end{align}

Итог:
$$yu_{\eta\eta} = 0$$

\subsection{Случай 3. Уравнение элиптического типа}
$$D < 0, -4xy < 0, x>0, y>0$$
$$dy = \sqrt{\frac{x}{y}}idx$$
$$\xi = 2/3y^{3/2}, \xi_x = 0, \xi_y = \sqrt{y}$$
$$\eta = 2/3x^{3/2}, \eta_x = \sqrt{x}, \eta_y = 0$$

\begin{align}
u_x &= \sqrt{x}u_\eta \notag \\
u_y &= \sqrt{y}u_\xi \notag \\
u_{xx} &= -\frac{1}{2\sqrt{x}}u_\eta + xu_{\eta\eta} \notag \\
u_{xy} &= \sqrt{x}\sqrt{y}u_{\xi\eta} \notag \\
u_{yy} &= \frac{1}{2\sqrt{y}}u_\xi + yu_{\xi\xi} \notag
\end{align}

Вычислим коэффициенты
\begin{align}
u_{\xi\xi} &: xy \notag     &u_{\xi\eta} : 0 \notag \\
u_{\eta\eta} &: xy \notag \\
u_\xi &: \frac{x}{2\sqrt{y}}; &u_\eta : \frac{y}{2\sqrt{x}} \notag 
\end{align}

Итог:
$$xyu_{\xi\xi} + xyu_{\eta\eta} + \frac{x}{2\sqrt{y}}u_\xi + \frac{y}{2\sqrt{x}}u_\eta$$

\section{Задание 4. Привести уравнение к каноническому виду и упростить}
$$u_{xx} - u_{yy} + u_x + u_y - 4u = 0$$
Характеристическое уравнение:
$$(dy)^2 - (dx)^2 = 0$$
$D = 4(dx)^2 > 0 \Rightarrow$ уравнение гиперболического типа.
Нетрудно видеть, что: 
\begin{align}
\xi &= y-x &\eta &= y+x \notag \\
\xi_x &= -1 &\eta_x &= 1 \notag \\
\xi_y &= 1 &\eta_y &= 1 \notag 
\end{align}

\begin{align}
u_x &= -u_\xi + u_\eta \notag \\
u_y &= u_\xi + u_\eta \notag \\
u_{xx} &= u_{\xi\xi} - 2u_{\xi\eta} + u_{\eta\eta} \notag \\
u_{xy} &= -u_{\xi\xi} + u_{\eta\eta} \notag \\
u_{yy} &= u_{\xi\xi} + 2u_{\xi\eta} + u_{\eta\eta} \notag
\end{align}

Вычислим коэффициенты
\begin{align}
    u_{\xi\xi} &: 0 &u_\xi &: 0 \notag \\
    u_{\xi\eta} &: -4 &u_\eta &: 2 \notag \\
    u_{\eta\eta} &: 0 \notag
\end{align}

Итог
$$2u_{\xi\eta} - u_\eta + 2u = 0$$

\section{Задание 5. Найти общее решение уравнения}
$$u_{xx} + 10u_{xy} + 24u_{yy} + u_x + 4u_y = y-4x$$
Характеристическое уравнение
$$(dy)^2 - 10dxdy + 24(dx)^2 = 0$$
$D = 4(dx)^2 > 0 \Rightarrow$ уравнение гиперболического типа
\begin{align}
dy &= 6dx &dy = 4dx \notag
\end{align}

\begin{align}
\xi &= y - 6x &\eta &= y-4x \notag \\
\xi_x &= -6 &\eta_x &= -4 \notag \\
\xi_y &= 1 &\eta_y &= 1 \notag
\end{align}

\begin{align}
    u_x &= -6u_\xi - 4u_\eta \notag \\
    u_y &= u_\xi + u_\eta \notag \\
    u_{xx} &= 36u_{\xi\xi} + 48u_{\xi\eta} + 16u_{\eta\eta} \notag \\
    u_{xy} &= -6u_{\xi\xi} - 10u_{\xi\eta} + -4u_{\eta\eta} \notag \\
    u_{yy} &= u_{\xi\xi} + 2u_{\xi\eta} + u_{\eta\eta} \notag
\end{align}

Вычислим коэффициенты
\begin{align}
u_{\xi\xi} &: 0 &u_\xi &: -2 \notag \\
u_{\xi\eta} &: -4 &u_\eta &: 0 \notag \\
u_{\eta\eta} &: 0 \notag
\end{align}

Получим уравнение
$$-4u_{\xi\eta} - 2u_\xi = \eta$$

Проведем замену:
$v = u_\xi$
$$-4v_\eta - 2v \eta = \eta$$
Найдем общее решение однородного уравнения:
$$v_{oo} = C(\xi)e^{-\frac{\eta}{2}}$$

Методом Лагранжа найдем частное решение неоднородного уравнения:
$$v = -\frac{1}{2}\eta + C_2(\xi)e^{-\frac{\eta}{2}} - 4$$

Получим общее решение неоднородного уравнения:
$$v_{OH} = u_\xi = C(\xi)e^{-\frac{\eta}{2}} -\frac{1}{2}\eta + C_2(\xi)e^{-\frac{\eta}{2}} - 4$$

Интегрируем по $\xi$
$$u = C_5(\xi)e^{-\frac{\eta}{2}} - \frac{\xi\eta}{2} - 4\xi$$

Подставим x, y:
$$u(x,y) = C_5(y-6x)e^{-\frac{y-4x}{2}} - \frac{(y-6x)(y-4x)}{2} - 4(y-6x)$$

\section{Задание 6. В каждой из областей, где сохраняется тип уравнения, найти общее решение уравнения}
Задача 3.37
$$xu_{xx} - 4x^2u_{xy} + 4x^3u_{yy} + u_x - 4xu_y = x(y+x^2)$$

\subsection{Приведем к каноническому виду}
    Характеристическое уравнение:
    $$x(dy)^2 + 4x^2dxdy + 4x^3(dx)^2 = 0$$
    $D = 0 \Rightarrow$ уравнение параболического типа
    $$dy = -2xdx$$
    \begin{align}
        \xi &= y+x^2 &\xi_x &= 2x &\xi_y &= 1 \notag \\
        \eta &= x &\eta_x &= 1 &\eta_y &= 0 \notag  
    \end{align}

    \begin{align}
        u_x &= 2xu_\xi + u_\eta \notag \\
        u_\eta &= u_\xi \notag \\
        u_{xx} &= 4x^2u_{\xi\xi} + 4xu_{\xi\eta} + u_{\eta\eta} + 2u_\xi \notag \\
        u_{xy} &= 2xu_{\xi\xi} + u_{\xi\eta} \notag \\
        u_{yy} &= u_{\xi\xi} \notag
    \end{align}

    Вычислим коэффициенты:
    \begin{align}
    u_{\xi\xi} &: 4x^3 - 8x^3 + 4x^3 = 0 &u_\xi &: 2x+2x-4x = 0 \notag \\
    u_{\xi\eta} &: 4x^2 - 4x^2 = 0 &u_\eta &: 1 \notag \\
    u_{\eta\eta} &: x \notag
    \end{align}

    В результате:
    $$\eta u_{\eta\eta} + u_\eta = \xi\eta$$

    \subsection{Найдем общее решение}
    Проведем замену: $v = u_\eta$
    Имеем уравнение:
    $$\eta v_\eta + v = \xi\eta$$
    Найдем общее решение однородного:
    $$v_{OO} = \frac{C(\xi)}{\eta}$$
    Методом Лагранжа найдем частное решение неоднородного:
    $$v_\eta = \frac{C_\eta\eta - C}{\eta^2}$$
    $$C_\eta\eta = \xi\eta^2 \Rightarrow C_\eta = \xi\eta$$
    $$C = \frac{1}{2} \xi\eta^2 + C_2(\xi)$$
    Частное решение неоднородного:
    $$v = \frac{1}{2}\xi\eta + \frac{1}{\eta}C_2(\xi)$$
    Общее решение неоднородного:
    $$v = u_\eta = \frac{C_3(\xi)}{\eta} + \frac{1}{2}\xi\eta$$
    Проинтегрируем по $\eta$
    $$u = C_3(\xi)ln\eta + \frac{1}{4}\xi\eta^2$$
    Выразим через x,y:
    $$u(x,y) = C_3(y+x^2)ln(x) + \frac{1}{4}(y+x^2)x^2$$

\section{Задние 7. Решить задачу Гурса}
Задача 12.
\begin{equation*}
    \begin{cases}
        u_{xx} + 6u_{xy} + 8u_{yy} + u_x + u_y = 0,
        \\
        x > 0, y>0,
        
    \end{cases}
\end{equation*}
\end{document}